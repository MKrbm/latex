\documentclass{jarticle}

% \usepackage[twoside,top=25truemm,bottom=20truemm,inner=30truemm,outer=20truemm]{geometry}

\makeatletter
\def\mojiparline#1{
    \newcounter{mpl}
    \setcounter{mpl}{#1}
    \@tempdima=\linewidth
    \advance\@tempdima by-\value{mpl}zw
    \addtocounter{mpl}{-1}
    \divide\@tempdima by \value{mpl}
    \advance\kanjiskip by\@tempdima
    \advance\parindent by\@tempdima
}
\makeatother
\def\linesparpage#1{
    \baselineskip=\textheight
    \divide\baselineskip by #1
}
\usepackage{comment}
\usepackage{booktabs}
\usepackage{amsmath}
\usepackage{amssymb}
\usepackage[dvipdfmx]{graphicx}
\usepackage{ascmac}
\usepackage{tikz}
\usepackage{siunitx}
\usepackage{gensymb}
\usetikzlibrary{svg.path,arrows,chains,matrix,positioning,scopes,intersections,calc,through,trees, positioning}
\usepackage{circuitikz}
\usepackage{soul}

\title{基礎のニューラルネット}

\newcommand\vc[1]{\boldsymbol{#1}}
\newcommand\RR{\mathbb R}
\newcommand\logi[1]{\varsigma(#1)}
\newcommand\logid[1]{\varsigma'(#1)}

\newcommand\relu[1]{\mathrm{ReLU}(#1)}

\begin{document}
% 一行あたり文字数の指定
\mojiparline{40}
% 1ページあたり行数の指定
\linesparpage{38}

\maketitle

\tableofcontents

% \newpage
% \input body.tex

\subsection{深層化とバックプロパゲーション}
\label{sec:back-propagation}

\section{非階層型ニューラルネットモデル}
\label{sec:non-hierarchical}

\section{深層学習へのいざない}
\label{sec:deep-learning}

\subsection{深層学習を支える技術}
\label{sec:basis-deep-learning}

\subsection{深層学習の応用}
\label{sec:application-of-deep-learning}

\subsection{transformer}
\input transformer.tex
\end{document}